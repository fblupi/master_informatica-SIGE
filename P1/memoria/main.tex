\input{estilo.tex}
\usepackage{textcomp}
\usepackage{hyperref}

%----------------------------------------------------------------------------------------
%	DATOS
%----------------------------------------------------------------------------------------

\newcommand{\myName}{Francisco Javier Bolívar Lupiáñez}
\newcommand{\myDegree}{Máster en Ingeniería Informática}
\newcommand{\myFaculty}{E. T. S. de Ingenierías Informática y de Telecomunicación}
\newcommand{\myDepartment}{Ciencias de la Computación e Inteligencia Artificial}
\newcommand{\myUniversity}{\protect{Universidad de Granada}}
\newcommand{\myLocation}{Granada}
\newcommand{\myTime}{\today}
\newcommand{\myTitle}{Práctica 1}
\newcommand{\mySubtitle}{Competición en Kaggle sobre Clasificación Binaria}
\newcommand{\mySubject}{Sistemas Inteligentes para la Gestión de la Empresa}
\newcommand{\myYear}{2016-2017}

%----------------------------------------------------------------------------------------
%	PORTADA
%----------------------------------------------------------------------------------------


\title{	
	\normalfont \normalsize 
	\textsc{\textbf{\mySubject \space (\myYear)} \\ \myDepartment} \\[20pt] % Your university, school and/or department name(s)
	\textsc{\myDegree \\[10pt] \myFaculty \\ \myUniversity} \\[25pt]
	\horrule{0.5pt} \\[0.4cm] % Thin top horizontal rule
	\huge \myTitle: \mySubtitle \\ % The assignment title
	\horrule{2pt} \\[0.5cm] % Thick bottom horizontal rule
	\normalfont \normalsize
}

\author{
	\myName \\ 
	\texttt{fblupi} \\
	\small Posición: X, Puntuación: 0.841
}

\date{\myTime} % Incluye la fecha actual
%----------------------------------------------------------------------------------------
%	INDICE
%----------------------------------------------------------------------------------------

\begin{document}
	
\definecolor{light-gray}{gray}{0.95}
	
\lstset {
	basicstyle=\scriptsize,
	frame=single,
	backgroundcolor=\color{grey}
}

\lstdefinestyle{R}{
	frame=single,
	numbers=left,
	language=R,
	basicstyle=\footnotesize,
	keywordstyle=\bfseries,
	commentstyle=\itshape,
	identifierstyle=\bfseries,
}
	
\setcounter{page}{0}

\maketitle % Muestra el Título
\thispagestyle{empty}

\newpage %inserta un salto de página

\tableofcontents % para generar el índice de contenidos

\listoftables

%\listoffigures

\newpage

%----------------------------------------------------------------------------------------
%	DOCUMENTO
%----------------------------------------------------------------------------------------

\section{Exploración de datos}

La primera fase que hay que realizar ante cualquier problema de este tipo es la exploración de datos para poner los cimientos sobre los que realizar el posterior proceso de análisis.
\\ \\
Los datos con los que contamos en el \textit{dataset} son \cite{KaggleTitanicData}:

\begin{itemize}
	\item \texttt{PassengerId}: Id único, no va a ser de utilidad
	\item \texttt{PClass}: Clase en la que viajaba
	\item \texttt{Name}: Nombre y título
	\item \texttt{Sex}: Sexo
	\item \texttt{Age}: Edad
	\item \texttt{SibSp}: Número de hermanos y/o cónyuges
	\item \texttt{Parch}: Número de padres y/o hijos
	\item \texttt{Ticket}: Id del billete
	\item \texttt{Fare}: Tarifa del viaje
	\item \texttt{Cabin}: Cabina donde se alojó
	\item \texttt{Embarked}: Dónde embarcó
\end{itemize}

De aquí podemos concluir que ni \texttt{PassengerId}, ni \texttt{Ticket} nos van a ser útiles pues son únicos. \texttt{Name} podría tener el mismo problema, pero se podría extraer el título para utilizarlo como información que puede resultar útil.
\\ \\
También hay otras variables como \texttt{Cabin} que son poco útiles pues tiene la mayoría de los valores perdidos.

\subsection{La importancia del sexo, la edad y la clase}

Se sabe que el protocolo por aquel entonces era el de mujeres y niños primero por lo que vamos a ver si de verdad se cumplió y nos puede ayudar a predecir quién murió y quién no.
\\ \\
En el \textit{dataset} de \textit{training}, el 62\% de los pasajeros murieron y frente al 38\% que sobrevivieron. En cuanto a hombres o mujeres, el 65\% eran hombres y el 35\% mujeres. Y en cuanto a niños o adultos (considerado adulto a partir de los 18 años), el 87\% eran adultos y el 13\% niños.
\\ \\
Combinando si son hombres o mujeres y adultos o niños tenemos la siguiente tabla:

En primer lugar, vamos a ver el porcentaje de supervivientes:

\begin{table}[H]
	\centering
	\caption{Porcentaje de supervivientes}
	\label{tab:die-survive}
	\begin{tabular}{|ll|}
		\hline
		Die       & Survive   \\ \hline
		0.6161616 & 0.3838384 \\ \hline
	\end{tabular}
\end{table}

de sexo:

\begin{table}[H]
	\centering
	\caption{Porcentaje de sexo}
	\label{tab:male-female}
	\begin{tabular}{|ll|}
		\hline
		Male     & Female   \\ \hline
		0.647587 & 0.352413 \\ \hline
	\end{tabular}
\end{table}

y de adultos (considerándose a partir de los 18 años):

\begin{table}[H]
	\centering
	\caption{Porcentaje de edad}
	\label{tab:child-adult}
	\begin{tabular}{|ll|}
		\hline
		Child     & Adult     \\ \hline
		0.1268238 & 0.8731762 \\ \hline
	\end{tabular}
\end{table}

Combinando edad y sexo tenemos el siguiente resultado:

\begin{table}[H]
	\centering
	\caption{Supervivencia según edad y sexo}
	\label{tab:age-sex}
	\begin{tabular}{|ll|l|}
		\hline
		Age   & Sex    & Survived  \\ \hline
		Adult & Female & 0.7528958 \\
		Child & Female & 0.6909091 \\
		Adult & Male   & 0.1657033 \\
		Child & Male   & 0.3965517 \\ \hline
	\end{tabular}
\end{table}

con el que podríamos hacer un primer envío dando por supervivientes a todas las mujeres. Solo con eso se tendría una puntuación de 0.76555.
\\ \\
No obstante, podemos intentar afinar más pues vemos que los hombres que son niños casi llega al 50\% de posibilidades de supervivencia. Agregando otros datos importantes como la clase y la tarifa del viaje obtenemos:

\begin{table}[H]
	\centering
	\caption{Supervivencia según edad, sexo, clase y tarifa}
	\label{tab:age-sex-fare-pclass}
	\begin{tabular}{|llll|l|}
		\hline
		Age   & Fare        & Pclass & Sex    & Survived   \\ \hline
		Adult & 20-30       & 1      & Female & 0.83333333 \\
		Adult & 30+         & 1      & Female & 0.98750000 \\
		Child & 30+         & 1      & Female & 0.87500000 \\
		Adult & 10-20       & 2      & Female & 0.90625000 \\
		Child & 10-20       & 2      & Female & 1.00000000 \\
		Adult & 20-30       & 2      & Female & 0.88000000 \\
		Child & 20-30       & 2      & Female & 1.00000000 \\
		Adult & 30+         & 2      & Female & 1.00000000 \\
		Child & 30+         & 2      & Female & 1.00000000 \\
		Adult & \textless10 & 3      & Female & 0.56140351 \\
		Child & \textless10 & 3      & Female & 0.85714286 \\
		Adult & 10-20       & 3      & Female & 0.50000000 \\
		Child & 10-20       & 3      & Female & 0.73333333 \\
		Adult & 20-30       & 3      & Female & 0.40000000 \\
		Child & 20-30       & 3      & Female & 0.16666667 \\
		Adult & 30+         & 3      & Female & 0.11111111 \\
		Child & 30+         & 3      & Female & 0.14285714 \\
		Adult & \textless10 & 1      & Male   & 0.00000000 \\
		Adult & 20-30       & 1      & Male   & 0.40000000 \\
		Adult & 30+         & 1      & Male   & 0.35365854 \\
		Child & 30+         & 1      & Male   & 1.00000000 \\
		Adult & \textless10 & 2      & Male   & 0.00000000 \\
		Adult & 10-20       & 2      & Male   & 0.11864407 \\
		Child & 10-20       & 2      & Male   & 0.75000000 \\
		Adult & 20-30       & 2      & Male   & 0.04761905 \\
		Child & 20-30       & 2      & Male   & 0.75000000 \\
		Adult & 30+         & 2      & Male   & 0.00000000 \\
		Child & 30+         & 2      & Male   & 1.00000000 \\
		Adult & \textless10 & 3      & Male   & 0.10931174 \\
		Child & \textless10 & 3      & Male   & 0.15384615 \\
		Adult & 10-20       & 3      & Male   & 0.12903226 \\
		Child & 10-20       & 3      & Male   & 0.71428571 \\
		Adult & 20-30       & 3      & Male   & 0.07142857 \\
		Child & 20-30       & 3      & Male   & 0.20000000 \\
		Adult & 30+         & 3      & Male   & 0.41666667 \\
		Child & 30+         & 3      & Male   & 0.07692308 \\ \hline
	\end{tabular}
\end{table}

donde podríamos afinar diciendo que si las mujeres son de la clase 3 y su tarifa es mayor a 20 murieron y los niños con tarifa mayor a 30 de clase 1 o 2 sobrevivieron. Con este pequeño árbol de decisión hecho manualmente explorando los datos obtenemos una puntuación mejor que la anterior de un total de 0.77990.

%----------------------------------------------------------------------------------------
%	REFERENCIAS
%----------------------------------------------------------------------------------------

\newpage

\bibliography{referencias} %archivo referencias.bib que contiene las entradas 
\bibliographystyle{plain} % hay varias formas de citar

\end{document}
\input{estilo.tex}
\usepackage{textcomp}
\usepackage{hyperref}
\usepackage{longtable}

%----------------------------------------------------------------------------------------
%	DATOS
%----------------------------------------------------------------------------------------

\newcommand{\myName}{Francisco Javier Bolívar Lupiáñez}
\newcommand{\myColleageName}{Juan Pablo González Casado}
\newcommand{\myDegree}{Máster en Ingeniería Informática}
\newcommand{\myFaculty}{E. T. S. de Ingenierías Informática y de Telecomunicación}
\newcommand{\myDepartment}{Ciencias de la Computación e Inteligencia Artificial}
\newcommand{\myUniversity}{\protect{Universidad de Granada}}
\newcommand{\myLocation}{Granada}
\newcommand{\myTime}{25 de junio de 2017}
\newcommand{\myTitle}{Práctica 2}
\newcommand{\mySubtitle}{Clasificación de Imágenes}
\newcommand{\mySubject}{Sistemas Inteligentes para la Gestión de la Empresa}
\newcommand{\myYear}{2016-2017}

%----------------------------------------------------------------------------------------
%	PORTADA
%----------------------------------------------------------------------------------------


\title{	
	\normalfont \normalsize 
	\textsc{\textbf{\mySubject \space (\myYear)} \\ \myDepartment} \\[20pt] % Your university, school and/or department name(s)
	\textsc{\myDegree \\[10pt] \myFaculty \\ \myUniversity} \\[25pt]
	\horrule{0.5pt} \\[0.4cm] % Thin top horizontal rule
	\huge \myTitle: \mySubtitle \\ % The assignment title
	\horrule{2pt} \\[0.5cm] % Thick bottom horizontal rule
	\normalfont \normalsize
}

\author{
	\myName \\ 
	\myColleageName \\ \\ 
	\small Kaggle: \texttt{EchaEquipos} \\
	\small Posición: 195, Puntuación: 0.80378 \\ 
}

\date{\myTime} % Incluye la fecha actual
%----------------------------------------------------------------------------------------
%	INDICE
%----------------------------------------------------------------------------------------

\begin{document}
	
\definecolor{light-gray}{gray}{0.95}
	
\lstset {
	basicstyle=\scriptsize,
	frame=single,
	backgroundcolor=\color{grey}
}

\lstdefinestyle{R}{
	frame=single,
	numbers=left,
	language=R,
	basicstyle=\tiny\ttfamily,
	keywordstyle=\bfseries,
	commentstyle=\itshape,
	identifierstyle=\bfseries,
}
	
\setcounter{page}{0}

\maketitle % Muestra el Título
\thispagestyle{empty}

\newpage %inserta un salto de página

\tableofcontents % para generar el índice de contenidos

%\listoftables
%\listoffigures

\newpage

%----------------------------------------------------------------------------------------
%	DOCUMENTO
%----------------------------------------------------------------------------------------

\section{Exploración de datos}

Los datos que tenemos son una serie de imágenes a color con un tamaño de 3096 x 4128 y un peso aproximado que ronda de entre 2.5 a 7.5 MB.
\\ \\
Estas imágenes son fotos de cérvix femenina (parte inferior del útero) echadas desde distintas distancias y ángulos.
\\ \\
En el caso de las imágenes de \textit{train} las tenemos estructuradas en tres directorios distintos (\texttt{Type\_1}, \texttt{Type\_2} y \texttt{Type\_3}), uno por cada clase. En el caso de las imágenes de \textit{test} las tenemos todas en un solo directorio pues desconocemos el tipo de cada uno y es que ese será el objetivo de la práctica: clasificar cada una de estas 512 imágenes en los tres tipos distintos.
\\ \\
El conjunto de datos de \textit{train} consta 1481 imágenes, no obstante Kaggle proporciona más imágenes adicionales para aumenta a 7004 el número de imágenes para entrenar nuestros modelos.

\subsection{\textit{Dataset} no balanceado}

Lo primero que podemos observar es que tanto el conjunto básico (Figura \ref{fig:num-images-train-dataset}) como el que incluye las imágenes adicionales (Figura \ref{fig:num-images-train-extra-dataset}) se encuentran desequilibrados con pocas imágenes del tipo 1 y muchas de los tipos 2 y 3. Esto es algo que tendremos que tener muy en cuenta a la hora de realizar el preprocesamiento de datos.

\begin{figure}[H]
	\centering
	\includegraphics[width=12cm]{img/num-images-train-dataset}
	\caption{Porcentaje de imágenes de cada tipo para el \textit{dataset} básico}
	\label{fig:num-images-train-dataset}
\end{figure}

\begin{figure}[H]
	\centering
	\includegraphics[width=12cm]{img/num-images-train-extra-dataset}
	\caption{Porcentaje de imágenes de cada tipo para el \textit{dataset} que incluye las imágenes adicionales}
	\label{fig:num-images-train-extra-dataset}
\end{figure}

\subsection{Imágenes que no corresponden a un cérvix}

Además, a la hora de explorar cada uno de estos datos, nos encontramos con imágenes que no correspondían a un cérvix o se encontraban demasiado borrosas (Figura \ref{fig:no-cervix-image}). En Kaggle hay un foro de discusión sobre esto \cite{ImagesExcluded}. Obviamente, en el posterior proceso de preprocesamiento de datos se eliminarán para que no agreguen ningún tipo de ruido a nuestro modelo.

\begin{figure}[H]
	\centering
	\includegraphics[width=3.5cm]{img/3086}
	\includegraphics[width=3.5cm]{img/4065}
	\includegraphics[width=3.5cm]{img/4533}
	\includegraphics[width=3.5cm]{img/4367}
	\caption{Ejemplos de imágenes que se encuentran en el \textit{dataset} pero no corresponden a un cérvix o se encuentran demasiado borrosas}
	\label{fig:no-cervix-image}
\end{figure}

\section{Preprocesamiento de datos}

Como comenté en la sección anterior. Lo primero que hicimos fue borrar todas las fotos que no correspondían a cérvix o estaban demasiado borrosas. Después de esto nos planteamos realizar \textit{data augmentation}. Para ello realizamos varios \textit{scripts} en R con los que realizábamos operaciones de rotación y volteo para generar hasta ocho imágenes más por cada imagen. No obstante no llegamos a utilizarlos por problemas en memoria en nuestros equipos, y es que ni siquiera podíamos leer a un tamaño de 224 x 224 (necesario para \textit{fine-tuning}) las 7000 imágenes de Kaggle.
\\ \\
Como hacía falta reducir el número de imágenes se aprovechó para hacer \textit{undersampling} y balancear el conjunto de datos (Figura \ref{fig:num-images-train-extra-balanced-dataset}). No se equilibró totalmente pero se hizo lo suficiente como para que dejase de ser un conjunto de datos desequilibrado.

\begin{figure}[H]
	\centering
	\includegraphics[width=12cm]{img/num-images-train-extra-balanced-dataset}
	\caption{Porcentaje de imágenes de cada tipo para el \textit{dataset} que incluye las imágenes adicionales al que posteriormente se le ha realizado \textit{undersampling} para balancearlo.}
	\label{fig:num-images-train-extra-balanced-dataset}
\end{figure}

El conjunto de datos final cuenta con 4716 imágenes (1217 del primer tipo, 1780 del segundo y 1719 del tercero).

\section{Técnicas de clasificación}

\subsection{\textit{Learning from scratch}}

La primera solución al problema la realizamos con \textit{learning from scratch}. Esto quiere decir, crear una CNN (\textit{Convolutional Neural Network}) desde cero. Para ello se puede utilizar un amplio abanico de herramientas. Pero el que finalmente utilizamos fue \textit{Keras} una librería que corre con \textit{Tensorflow} como \textit{backend} en Python.
\\ \\
Comenzamos siguiendo un \textit{kernel} \cite{StartKernel} con el que se garantizaba bajar el \textit{loss} de 1.
\\ \\
Este \textit{kernel} redimensiona y normaliza las imágenes, realiza \textit{data augmentation}, genera un conjunto de validación a partir de los datos de \textit{train} y entrena una red neuronal simple (Figura \ref{fig:kernel-cnn}) con la que posteriormente predice y exporta a CSV.

\begin{figure}[H]
	\centering
	\includegraphics[height=18cm]{img/kernel-cnn}
	\caption{Topología de la red usada en el \textit{kernel} básico que se ha utilizado como base.}
	\label{fig:kernel-cnn}
\end{figure}

Probamos con distintos parámetros y distintas topologías cada vez más complejas pero no conseguíamos mejorar los resultados obtenidos con topologías más simples. ¿A qué se debe esto? Quizás a que no se dejó entrenar durante un buen número de épocas. Pero sobre todo a que estábamos creando una topología a ciegas. Para perder horas de cómputo con esta red compleja, ¿por qué no probar con una topología que ha dado buenos resultados en otros problemas?
\\ \\
Era hora de pasar a probar el \textit{fine-tuning}.

\subsection{\textit{Fine-tuning}}

La gran ventaja con \textit{fine-tuning} con respecto a \textit{learning from scratch} es que se utiliza una topología para la CNN ya entrenada para un problema en el que ha tenido éxito. Por lo que sabes que estás utilizando una topología con la que obtener un resultado medianamente bueno si la ajustas bien a tu problema.
\\ \\
Y es que, ajustarla a un problema específico no tiene más que añadir una capa de salida con la salida que necesitamos. En la documentación de \textit{Keras} hay un ejemplo de cómo hacerlo \cite{KerasApplications}.

\subsection{\textit{OVO} y \textit{OVA}}

Hasta ahora se ha realizado clasificación multiclase en la que la red neuronal tenía una salida \textit{softmax} con tres nodos, uno correspondiente a cada clase.
\\ \\
El siguiente enfoque es probar otras dos técnicas distintas a la hora de clasificar utilizando las mismas CNN. Estas dos técnicas tratan de convertir el problema de clasificación multiclase en problemas de clasificación binarios y son One vs. One (OVO) y One vs. All (OVA), también conocido como One vs. Rest (OVR).

\subsubsection{\textit{OVO}}

Usando OVO se entrenarán tres clasificadores distintos para distinguir entre dos clases: Tipo 1 vs. Tipo 2, Tipo 1 vs. Tipo 3 y Tipo 2 vs. Tipo 3.
\\ \\
Obviamente, para cada clasificador se utilizarán como entrenamiento las imágenes de los tipos que se comparan, ignorando la del tipo que no está comparando.
\\ \\
Una vez entrenados se va a predecir el tipo de cada imagen pasándola por cada clasificador que obtendrá como salida dos porcentajes, uno por cada uno de los dos tipos. Estos porcentajes se combinarán para obtener el resultado final de porcentaje para cada uno de los tres tipos.
\\ \\
Para combinar estos datos no hay ningún método preestablecido y se podrían utilizar varias técnicas. Por ejemplo, supongamos que tenemos estas salidas:

\begin{itemize}
	\item Tipo 1: 0,7 vs Tipo 2: 0,3
	\item Tipo 1: 0,55 vs Tipo 3: 0,45
	\item Tipo 2: 0,4 vs Tipo 3: 0,6
\end{itemize}

Se podría multiplicar las dos salidas de cada tipo y normalizarlas para que la suma de los tres porcentajes sea 1.

\begin{itemize}
	\item Combinación tipo 1: $ 0,7 \times 0,55 = 0,385 $
	\item Combinación tipo 2: $ 0,3 \times 0,4 = 0,12 $
	\item Combinación tipo 3: $ 0,45 \times 0,6 = 0,27 $
	\item Tipo 1 + Tipo 2 + Tipo 3 = $ 0,385 + 0,12 + 0,27 = 0,775 $
	\item Salida tipo 1: $ \frac{0,385}{0,775} = 0.5 $
	\item Salida tipo 2: $ \frac{0,12}{0,775} = 0.15 $
	\item Salida tipo 3: $ \frac{0,27}{0,775} = 0.35 $
\end{itemize}

Otra opción podría ser realizar la media que directamente obtiene una salida normalizada:

\begin{itemize}
	\item Salida tipo 1: $ \frac{0,7 + 0,55}{3} = 0.42 $
	\item Salida tipo 2: $ \frac{0,3 + 0,4}{3} = 0.23 $
	\item Salida tipo 3: $ \frac{0,45 + 0,6}{3} = 0.35 $
\end{itemize}

\subsubsection{\textit{OVA}}

Usando OVA se utilizarán también otros tres clasificadores pero esta vez distinguirán entre una clase y el resto: Tipo 1 vs. Tipo 2-3, Tipo 2 vs. Tipo 1-3 y Tipo 3 vs. Tipo 1-2.
\\ \\
Para ello se pueden crear tres directorios nuevos combinando las imágenes de dos de los tipos y crear los clasificadores igual que con OVO. 
\\ \\
Para combinar los resultados tampoco hay ningún método estipulado. El método más sencillo que se nos ha ocurrido es utilizar las salidas en las que el tipo hace de \textit{one} y normalizarlas para que sus sumas sean 1.
\\ \\
Por ejemplo, supongamos que tenemos estas salidas en los distintos clasificadores:

\begin{itemize}
	\item Tipo 1: 0,55 vs Tipo 2-3: 0,45
	\item Tipo 2: 0,4 vs Tipo 1-3: 0,6
	\item Tipo 3: 0,1 vs Tipo 1-2: 0,9
\end{itemize}

La salida final sería:

\begin{itemize}
	\item Salida tipo 1: $ \frac{0,55}{0,55 + 0,4 + 0,1} = 0.52 $
	\item Salida tipo 2: $ \frac{0,4}{0,55 + 0,4 + 0,1} = 0.38 $
	\item Salida tipo 3: $ \frac{0,1}{0,55 + 0,4 + 0,1} = 0.1 $
\end{itemize}

\subsection{Extracción de características}

Hasta ahora, ya sea en \textit{learning from scratch} o \textit{fine-tuning}, o en clasificación multiclase, OVO u OVA, la técnica utilizada para predecir ha sido una CNN. No obstante hay más posibilidades para predecir clases.
\\ \\
Una de ellas es la extracción de características extrayendo los mapas de características de la CNN para su posterior uso en técnicas clásicas de \textit{machine learning} como \textit{random forest}, \textit{boosting} o SVM (\textit{Support Vector Machine}).
\\ \\
Para la extraer las características de una CNN usando \textit{Keras} se puede seguir el ejemplo que hay en su documentación \cite{KerasApplications}. Y es que se haría igual que como se ha venido haciendo hasta ahora pero definiendo como salida del modelo a la hora de predecir la capa donde se encuentran los mapas de características.
\\ \\
Una vez obtenemos los mapas de características podemos usar la librería \textit{scikit-learn} \cite{Sklearn} donde encontramos distintas técnicas como el \textit{random forest} y el SVM \cite{KerasAndSklearn}.

\subsection{\textit{Ensemblers}}

Por último, se podría realizar un \textit{ensembler} con el que combinar varias de las técnicas citadas anteriormente con el objetivo de obtener un mejor resultado.

\section{Presentación y discusión de resultados}

De las técnicas mencionadas anteriormente solo se pudieron probar exhaustivamente las dos primeras (\textit{learning from scratch} y \textit{fine-tuning}) debido a la falta de tiempo.

\subsection{\textit{Learnin from scratch}}

Lo primero que hicimos fue crear una CNN  desde cero. Para ello intentamos realizarlo con \textit{MXNet} en R. Pero tras varios días lanzando el \textit{script} con distintos parámetros no conseguíamos que predijese porcentajes distintos para cada imagen. Y es que al principio pensábamos que era por culpa de la red simple que usábamos pero al probar con una topología famosa (AlexNet) y seguir obteniendo estos resultados, empezamos a sospechar que era algún error de programación en el \textit{script}. Lo depuramos para intentar encontrar el \textit{bug}, pero al no encontrarlo pasamos a utilizar \textit{Keras} en Python.
\\ \\
Se empezó utilizando el conjunto de datos de \textit{train} completo (pues todavía no se había realizado el \textit{undersampling} para el balanceo).
\\ \\
Se probó a variar el número de épocas para ver cuándo se estancaba la pérdida de \textit{loss} (o subía) en el conjunto de validación para detectar cuando sobre aprendía (Figura \ref{fig:loss-variation-learning-from-scratch}).

\begin{figure}[H]
	\centering
	\includegraphics[width=12cm]{img/loss-variation-learning-from-scratch}
	\caption{Variación de \textit{log loss} en los distintos conjuntos de datos (\textit{train} y \textit{test}) a lo largo de las épocas.}
	\label{fig:loss-variation-learning-from-scratch}
\end{figure}

En el gráfico (Figura \ref{fig:loss-variation-learning-from-scratch}) se ha muestreado en las iteraciones 10, 20 y 50 en una CNN simple usando imágenes de 128x128 con un tamaño de \textit{batch} de 15. Se puede ver como en la época 50 ya ha aprendido y como en la 20 se obtienen mejores resultados que en la 10. Se podrían haber hecho más experimentos hasta encontrar la época exacta donde empieza a sobre aprender, pero nos sirvió como primera aproximación para saber más o menos cuántas épocas utilizar.
\\ \\
También se probó a variar el tamaño de las imágenes y vimos como con una red tan simple apenas se notaba la diferencia y se podían usar imágenes de incluso 32x32 obteniendo resultados similares a los obtenidos con imágenes de 128x128.
\\ \\
Lo siguiente que se probó a variar fue el tamaño del \textit{batch} y el \textit{samples per epoch}. Viendo que si se aumentaba este segundo parámetro se llegaba a aprender en menor número de épocas.
\\ \\
Por último se probó a hacer más compleja la red neuronal agregando dos ciclos más de capas \textit{Convolution}, \textit{Activation}, \textit{Pooling}, así como agregar más capas \textit{Fully Connected}. Sin embargo lo único que se consiguió con esto fue aumentar el tiempo de cómputo, que era casi despreciable con el modelo sencillo, porque los resultados eran ligeramente peores que los obtenidos anteriormente con la CNN simple.
\\ \\
El mejor resultado obtenido con esta técnica fue de 0,87668	usando imágenes de 128x128 (a las que se hizo un \textit{undersampling} balanceando las clases) durante 10 épocas usando un tamaño de \textit{batch} pequeño (15). Y es que el utilizar el conjunto de datos de \textit{train} balanceado consiguió que mejorásemos nuestro resultado bastante desde el anterior mejor resultado de 0,91442.

\subsection{\textit{Fine-tuning}}

Con el \textit{fine-tuning} nos encontramos muchísimos problemas. No en la implementación que se encuentra publicada en la propia web de \textit{Keras} \cite{KerasApplications}.
\\ \\
Estos problemas eran relacionados con las limitaciones \textit{hardware} de nuestros equipos.
\\ \\
El primero de ellos nos impedía entrenar la CNN y es que no teníamos memoria suficiente para alojar a un tamaño de 224x224 (necesario para la mayoría de las redes) las 7000 fotos que proporciona Kaggle. Es entonces cuando aprovechamos para hacer el \textit{undersampling} y consecuente balanceo del \textit{dataset}.
\\ \\
El segundo problema era el del tiempo de cómputo. Pues al no contar con GPU una sola época duraba entre 40 y 80 minutos dependiendo de la topología. Lo que nos impedía realizar experimentos con agilidad. Por ello nos llevó varios días dejando entrenar la CNN hasta 13 horas. Por suerte, los dos últimos días se pudo contar con una GPU que reducía enormemente estos tiempos pasando de 40 minutos a 40 segundos por época. Pero era demasiado tarde y apenas se pudieron realizar unos pocos experimentos.
\\ \\
Se probó como base las topologías \textit{VGG16}, \textit{VGG19} y \textit{ResNet50} llegando a hacer \textit{submissions} solo con esta última pues no se consiguió que se aprendiese con las dos primeras. El \textit{log loss} era muy alto tras varias épocas incluso para el conjunto de datos de \textit{train}.
\\ \\
Se crearon tres topologías utilizando \textit{ResNet50} (Figura \ref{fig:fine-tuning-topologies}):

\begin{itemize}
	\item Una primera en la que se conectaba directamente la salida de ResNet50 a una capa \textit{Flatten}, una \textit{Fully Connected} de tamaño 3 (las tres posibles salidas) que se conecta con una activación \textit{softmax}.
	\item Otra utilizando la misma que se muestra en la web de \textit{Keras} \cite{KerasApplications}.
	\item Otra igual a la anterior pero disminuyendo los nodos de la capa \textit{Fully Connected}.
\end{itemize}

\begin{figure}[H]
	\centering
	\includegraphics[width=2cm]{img/fine-tuning-topology-1}
	\includegraphics[width=2cm]{img/fine-tuning-topology-2}
	\includegraphics[width=2cm]{img/fine-tuning-topology-3}
	\caption{Topologías utilizadas para el \textit{fine-tuning}.}
	\label{fig:fine-tuning-topologies}
\end{figure}

Con la primera topología no se consiguieron buenos resultados y con la segunda se veía como sobreaprendía muy rápidamente. Fue con la última de ellas con la que se consiguió mejorar los resultados obtenidos usando \textit{learning from scratch}.
\\ \\
Al igual que se hizo con \textit{learning from scratch} se probó a cambiar distintos parámetros hasta obtener mejores resultados.
\\ \\
Además se probó a realizar el entrenamiento solo sobre las capas añadidas manualmente tras \textit{ResNet50}, entrenando también algunas de la red importada o realizando un entrenamiento en dos fases. La primera (durante pocas épocas) con un ratio de aprendizaje mayor sobre las capas añadidas manualmente y la segunda con un ratio de aprendizaje menor también sobre las últimas capas de \textit{ResNet50}.
\\ \\
Se observó como el \textit{log loss} sobre el conjunto de \textit{train} disminuía continuamente hasta la época que se dejó mientras que el de validación se estancaba antes en 0,8. Con el conjunto de \textit{test}, al subir los resultados a Kaggle, se vio que empezaba a dar peores resultados antes (Figura \ref{fig:loss-variation-fine-tuning}).

\begin{figure}[H]
	\centering
	\includegraphics[width=12cm]{img/loss-variation-fine-tuning}
	\caption{Variación de \textit{log loss} en los distintos conjuntos de datos (\textit{train}, validación y \textit{test}) a lo largo de las épocas.}
	\label{fig:loss-variation-fine-tuning}
\end{figure}

En el gráfico (Figura \ref{fig:loss-variation-fine-tuning}) se muestrea en las épocas 50, 60, 65, 70 y 75. Obteniendo el mejor resultados en la 60. Sin embargo, no sabemos con certeza si entre la época 50 y 60 o la 60 y 65 hay un resultado mejor. Y es que elegir en qué época exacta parar de entrenar es complicado, aunque elegir una aproximada si puede resultar más sencillo viendo cómo se comporta el \textit{log loss} en los conjuntos de \textit{train} y validación pues nos pueden estar dando pistas de cuándo está sobreaprendiendo. En el caso de ejemplo que mostramos se ve cómo a partir de la época 65 aumenta el ritmo de bajada en el conjunto de \textit{train} y no en el de validación. A partir de la época 75 (no aparece en la gráfica) el \textit{log loss} sobre el conjunto de validación se estancaba llegando a aumentar en las épocas posteriores. Estos dos hechos nos hacen despreciar lo que pase a partir de estas épocas y centrarnos en las anteriores para buscar el mejor resultado posible.
\\ \\
El mejor resultado obtenido fue de 0,80378 que sirvió para colocarnos en el top 200 del ranking. Lo obtuvimos usando como base la red \textit{ResNet50} agregando una capa de 512 nodos entre la red original y nuestra salida. Se obtuvo con un tamaño de \textit{batch} de 48 y tras 20 épocas entrenando solo las capas agregadas por nosotros y 40 entrenando también las últimas capas del modelo original.

\section{Conclusiones y trabajo futuro}

Esta práctica ha tenido sus aspectos positivos y negativos. Empezando por los positivos, es bonito aplicar las técnicas vistas en clase a un problema real que nos podríamos encontrar en la calle y ver cómo acabamos en el top 200 de más de 800 equipos resulta gratificante.
\\ \\
Sin embargo, el tiempo que se ha empleado en esta práctica ha sido mayor al esperado y lo peor es que ni aún así se han podido realizar todos los objetivos que se habían planteado.
\\ \\
Esto ha sido debido a los altos requisitos \textit{hardware} que se necesitaban para procesar los datos del problema y una máquina normal podía entrenar una red neuronal pero en un período de tiempo enorme y es que la mayoría no disponemos de una tarjeta gráfica de alto rendimiento con la que acelerar los resultados.
\\ \\
En nuestro caso, uno de los dos no podía ni entrenar una sola red neuronal por los requisitos de memoria RAM que superaban los 4GB con los que disponía cuando se utilizaba un \textit{dataset} más amplio para entrenar, además de los problemas de sobrecalentamiento que impedían entrenar más de cinco épocas con un modelo simple en \textit{learning from scratch}.
\\ \\
Al menos sí disponíamos de un portátil con el que realizar estos cálculos que ha estado funcionando a pleno rendimiento durante varios días seguidos (sin descanso) llegando a entrenar redes neuronales durante 13 horas. Pero, también por problemas de memoria no hemos podido realizar el \textit{data augmentation} al nivel que queríamos y hemos tenido que hacer \textit{undersampling} en lugar de \textit{oversampling} para balancear el conjunto de datos de \textit{train}.
\\ \\
Cuando quedaban dos días para el cierre del plazo de envíos tuvimos la suerte de que nos prestasen una tarjeta gráfica con la que realizar los últimos experimentos y dada esta oportunidad, en vez de probar el resto de técnicas, preferimos invertir las \textit{submissions} restantes en mejorar nuestro mejor resultado utilizando \textit{fine tuning}, pues los experimentos realizados con anterioridad nos daban una idea de qué parámetros utilizar. Y es que, si hubiésemos tenido que realizar el segundo experimento que lanzamos en la tarjeta gráfica en la CPU que habíamos utilizado durante estas dos semanas habríamos necesitado 3 días y 8 horas en lugar de la hora y media que tardó en GPU.
\\ \\
En conclusión, ha sido una práctica que ha acumulado más momentos de desesperación que de ilusión y no por el problema en sí, sino por los tiempos de cómputo.
\\ \\
Como trabajo futuro, indudablemente, está el probar el resto de experimentos que no han dado tiempo a hacer. Convertir el problema multiclase en uno binario probando OVO-OVA y probar con técnicas de \textit{machine learning} clásicas con los mapas de características extraídos de una CNN.

\section{Listado de soluciones}

Se realizaron un total de 26 \textit{submissions} aunque una fue subiendo por segunda vez el mismo CSV por error. Por lo que aquí no se  muestra esa duplicada y hay 25.

\begin{longtable}{|
		p{\dimexpr0.05\textwidth-2\tabcolsep-\arrayrulewidth\relax}|
		p{\dimexpr0.28\textwidth-2\tabcolsep-\arrayrulewidth\relax}|
		p{\dimexpr0.15\textwidth-2\tabcolsep-\arrayrulewidth\relax}|
		p{\dimexpr0.135\textwidth-2\tabcolsep-\arrayrulewidth\relax}|
		p{\dimexpr0.135\textwidth-2\tabcolsep-\arrayrulewidth\relax}|
		p{\dimexpr0.115\textwidth-2\tabcolsep-\arrayrulewidth\relax}|
		p{\dimexpr0.135\textwidth-2\tabcolsep-\arrayrulewidth\relax}|
	}
	\hline
	\# & Descripción & Algoritmo y software empleado & Resultado (\textit{test}) & Resultado (\textit{train}) & Posición & Fecha y hora \\
	\hline
	\hline
	1 & CNN simple \textit{learning from scratch} con datos de \textit{train} básicos & R, \textit{MXNet} & 1,11456 & 1,06880 & 671 & 5/6/17 10:25 \\
	\hline 
	2 & CNN simple \textit{learning from scratch} con más datos de \textit{train} y balanceados & R, \textit{MXNet} & 1,06771 & 1,03348 & 646 & 7/6/17 11:17 \\
	\hline 
	3 & CNN simple \textit{learning from scratch} con más datos de \textit{train} (no balanceados) a 128x128 durante 10 épocas & Python, \textit{Keras} & 0,9991 & 0,95566 & 412 & 8/6/17 13:23 \\
	\hline 
	4 & CNN simple \textit{learning from scratch} con más datos de \textit{train} (no balanceados) a 32x32 durante 20 épocas & Python, \textit{Keras} & 0,91442 & 0,9575 & 344 & 8/6/17 17:32 \\
	\hline 
	5 & CNN simple \textit{learning from scratch} con más datos de \textit{train} (no balanceados) a 128x128 durante 50 épocas & Python, \textit{Keras} & 0,94097 & 0,8224 & 347 & 8/6/17 18:35 \\
	\hline 
	6 & CNN simple \textit{learning from scratch} con más datos de \textit{train} (no balanceados) a 128x128 durante 20 épocas & Python, \textit{Keras} & 0,91071 & 0,8804 & 342 & 8/6/17 19:16 \\
	\hline 
	7 & CNN bastante más compleja \textit{learning from scratch} con más datos de \textit{train} (no balanceados) a 128x128 durante 15 épocas con un \textit{batch size} muy grande & Python, \textit{Keras} & 0,99942 & 1.0444 & 344 & 8/6/17 22:50 \\
	\hline 
	8 & CNN simple \textit{learning from scratch} con más datos de \textit{train} (filtrados y algo más balanceados) a 128x128 durante 10 épocas con un \textit{batch size} muy pequeño & Python, \textit{Keras} & 0,87668 & 0,8261 & 282 & 9/6/17 11:23 \\
	\hline 
	9 & Primer intento con \textit{fine-tuning} en R & R, \textit{MXNet} & 2,24609 & 1,2871 & 282 & 9/6/17 11:34 \\
	\hline 
	10 & \textit{Fine-tuning} \textit{ResNet50} con capa \textit{fully connected} de tamaño 1024. Durante 5 épocas con un \textit{batch size} de 15 & Python, \textit{Keras} & 0,91451 & 0,6656 & 288 & 10/6/17 16:00 \\
	\hline 
	11 & \textit{Fine-tuning} \textit{ResNet50} con capa \textit{fully connected} de tamaño 512. Durante 10 épocas con un \textit{batch size} de 15 & Python, \textit{Keras} & 0,84681 & 0,7661 & 250 & 10/6/17 22:00 \\
	\hline 
	12 & \textit{Fine-tuning} \textit{ResNet50}. Durante 25 épocas con un \textit{batch size} de 50 y 2000 \textit{samples per epoch} & Python, \textit{Keras} & 1,11204 & 0,8318 & 251 & 11/6/17 9:03 \\
	\hline 
	13 & \textit{Fine-tuning} \textit{ResNet50} con capa \textit{fully connected} de tamaño 512. Durante 5 épocas con un \textit{batch size} de 15 & Python, \textit{Keras} & 0,89159 & 0,8275 & 251 & 11/6/17 15:15 \\
	\hline 
	14 & \textit{Fine-tuning} \textit{ResNet50} con capa \textit{fully connected} de tamaño 512. Durante 10 épocas con un \textit{batch size} de 15 & Python, \textit{Keras} & 0,90005 & 0,7875 & 251 & 12/6/17 15:15 \\
	\hline 
	15 & CNN simple \textit{learning from scratch} con más datos de \textit{train} balanceados a 128x128 durante 20 épocas con un \textit{batch size} muy pequeño & Python, \textit{Keras} & 0,99534 & 0,7298 & 255 & 12/6/17 18:13 \\
	\hline 
	16 & \textit{Fine-tuning} \textit{ResNet50}. Durante 50 épocas con un \textit{batch size} de 32 y 2048 \textit{samples per epoch} & Python, \textit{Keras} & 0,91639 & 0,7352 & 259 & 13/6/17 21:50 \\
	\hline 
	17 & \textit{Fine-tuning} \textit{ResNet50}. Durante 75 épocas con un \textit{batch size} de 32 y 2048 \textit{samples per epoch} & Python, \textit{Keras} & 0,90753 & 0,7513 & 259 & 13/6/17 21:51 \\
	\hline 
	18 & \textit{Fine-tuning} \textit{ResNet50}. Durante 100 épocas con un \textit{batch size} de 32 y 2048 \textit{samples per epoch} & Python, \textit{Keras} & 0,95160 & 0,7130 & 259 & 13/6/17 21:52 \\
	\hline 
	19 & \textit{Fine-tuning} \textit{ResNet50}. Durante 125 épocas con un \textit{batch size} de 32 y 2048 \textit{samples per epoch} & Python, \textit{Keras} & 0,96648 & 0,6980 & 259 & 13/6/17 21:53 \\
	\hline 
	20 & \textit{Fine-tuning} \textit{ResNet50}. Durante 75 épocas con un \textit{batch size} de 48 y 2048 \textit{samples per epoch} & Python, \textit{Keras} & 0,9170 & 0,7135 & 259 & 13/6/17 22:11 \\
	\hline 
	21 & \textit{Fine-tuning} \textit{ResNet50} con capa \textit{fully connected} de tamaño 512. Entrenamiento en dos pasos durante 20+55 épocas con un \textit{batch size} de 48 & Python, \textit{Keras} & 0,85543 & 0,3170 & 259 & 14/6/17 11:20 \\
	\hline 
	22 & \textit{Fine-tuning} \textit{ResNet50} con capa \textit{fully connected} de tamaño 512. Entrenamiento en dos pasos durante 20+50 épocas con un \textit{batch size} de 48 & Python, \textit{Keras} & 0,85133 & 0,3699 & 259 & 14/6/17 11:25 \\
	\hline 
	23 & \textit{Fine-tuning} \textit{ResNet50} con capa \textit{fully connected} de tamaño 512. Entrenamiento en dos pasos durante 20+30 épocas con un \textit{batch size} de 48 & Python, \textit{Keras} & 0,82368 & 0,5239 & 229 & 14/6/17 11:30 \\
	\hline 
	\textbf{24} & \textbf{\textit{Fine-tuning} \textit{ResNet50} con capa \textit{fully connected} de tamaño 512. Entrenamiento en dos pasos durante 20+40 épocas con un \textit{batch size} de 48} & \textbf{Python, \textit{Keras}} & \textbf{0,80378} & \textbf{0,4498} & \textbf{195} & \textbf{14/6/17 11:35} \\
	\hline 
	25 & \textit{Fine-tuning} \textit{ResNet50} con capa \textit{fully connected} de tamaño 512. Entrenamiento en dos pasos durante 20+45 épocas con un \textit{batch size} de 48 & Python, \textit{Keras} & 0,83104 & 0,4193 & 195 & 14/6/17 11:40 \\
	\hline
	\caption{Tabla cronológica con los resultados obtenidos}
\end{longtable}

%----------------------------------------------------------------------------------------
%	REFERENCIAS
%----------------------------------------------------------------------------------------

\newpage

\bibliography{referencias} %archivo referencias.bib que contiene las entradas 
\bibliographystyle{plain} % hay varias formas de citar

\end{document}
\input{estilo.tex}
\usepackage{textcomp}
\usepackage{hyperref}

%----------------------------------------------------------------------------------------
%	DATOS
%----------------------------------------------------------------------------------------

\newcommand{\myName}{Francisco Javier Bolívar Lupiáñez}
\newcommand{\myColleageName}{Juan Pablo González Casado}
\newcommand{\myDegree}{Máster en Ingeniería Informática}
\newcommand{\myFaculty}{E. T. S. de Ingenierías Informática y de Telecomunicación}
\newcommand{\myDepartment}{Ciencias de la Computación e Inteligencia Artificial}
\newcommand{\myUniversity}{\protect{Universidad de Granada}}
\newcommand{\myLocation}{Granada}
\newcommand{\myTime}{\today}
\newcommand{\myTitle}{Práctica 2}
\newcommand{\mySubtitle}{Clasificación de Imágenes}
\newcommand{\mySubject}{Sistemas Inteligentes para la Gestión de la Empresa}
\newcommand{\myYear}{2016-2017}

%----------------------------------------------------------------------------------------
%	PORTADA
%----------------------------------------------------------------------------------------


\title{	
	\normalfont \normalsize 
	\textsc{\textbf{\mySubject \space (\myYear)} \\ \myDepartment} \\[20pt] % Your university, school and/or department name(s)
	\textsc{\myDegree \\[10pt] \myFaculty \\ \myUniversity} \\[25pt]
	\horrule{0.5pt} \\[0.4cm] % Thin top horizontal rule
	\huge \myTitle: \mySubtitle \\ % The assignment title
	\horrule{2pt} \\[0.5cm] % Thick bottom horizontal rule
	\normalfont \normalsize
}

\author{
	\myName \\ 
	\myColleageName \\ \\ 
	\small Kaggle: \texttt{EchaEquipos} \\
	\small Posición: 195, Puntuación: 0.80378 \\ 
}

\date{\myTime} % Incluye la fecha actual
%----------------------------------------------------------------------------------------
%	INDICE
%----------------------------------------------------------------------------------------

\begin{document}
	
\definecolor{light-gray}{gray}{0.95}
	
\lstset {
	basicstyle=\scriptsize,
	frame=single,
	backgroundcolor=\color{grey}
}

\lstdefinestyle{R}{
	frame=single,
	numbers=left,
	language=R,
	basicstyle=\tiny\ttfamily,
	keywordstyle=\bfseries,
	commentstyle=\itshape,
	identifierstyle=\bfseries,
}
	
\setcounter{page}{0}

\maketitle % Muestra el Título
\thispagestyle{empty}

\newpage %inserta un salto de página

\tableofcontents % para generar el índice de contenidos

%\listoftables
%\listoffigures

\newpage

%----------------------------------------------------------------------------------------
%	DOCUMENTO
%----------------------------------------------------------------------------------------

\section{Exploración de datos}

Los datos que tenemos son una serie de imágenes a color con un tamaño de 3096 x 4128 y un peso aproximado que ronda de entre 2.5 a 7.5 MB.
\\ \\
Estas imágenes son fotos de cérvix femenina (parte inferior del útero) echadas desde distintas distancias y ángulos.
\\ \\
En el caso de las imágenes de \textit{train} las tenemos estructuradas en tres directorios distintos (\texttt{Type\_1}, \texttt{Type\_2} y \texttt{Type\_3}), uno por cada clase. En el caso de las imágenes de \textit{test} las tenemos todas en un solo directorio pues desconocemos el tipo de cada uno y es que ese será el objetivo de la práctica: clasificar cada una de estas 512 imágenes en los tres tipos distintos.
\\ \\
El conjunto de datos de \textit{train} consta 1481 imágenes, no obstante Kaggle proporciona más imágenes adicionales para aumenta a 7004 el número de imágenes para entrenar nuestros modelos.
\\ \\
Lo primero que podemos observar es que tanto el conjunto básico (Figura \ref{fig:num-images-train-dataset}) como el que incluye las imágenes adicionales (Figura \ref{fig:num-images-train-extra-dataset}) se encuentran desequilibrados con pocas imágenes del tipo 1 y muchas de los tipos 2 y 3. Esto es algo que tendremos que tener muy en cuenta a la hora de realizar el preprocesamiento de datos.

\begin{figure}[H]
	\centering
	\includegraphics[width=14cm]{img/num-images-train-dataset}
	\caption{Porcentaje de imágenes de cada tipo para el \textit{dataset} básico}
	\label{fig:num-images-train-dataset}
\end{figure}

\begin{figure}[H]
\centering
\includegraphics[width=14cm]{img/num-images-train-extra-dataset}
\caption{Porcentaje de imágenes de cada tipo para el \textit{dataset} que incluye las imágenes adicionales}
\label{fig:num-images-train-extra-dataset}
\end{figure}

\section{Preprocesamiento de datos}

Lala

\section{Técnicas de clasificación}

\subsection{\textit{Learning from scratch}}

Lala

\subsection{\textit{Fine-tuning}}

Lala

\section{Presentación y discusión de resultados}

Lala

\section{Conclusiones y trabajo futuro}

Lala

\section{Listado de soluciones}

Lala

%----------------------------------------------------------------------------------------
%	REFERENCIAS
%----------------------------------------------------------------------------------------

\newpage

\bibliography{referencias} %archivo referencias.bib que contiene las entradas 
\bibliographystyle{plain} % hay varias formas de citar

\end{document}